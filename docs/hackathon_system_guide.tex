\documentclass[11pt,a4paper]{article}
\usepackage[utf8]{inputenc}
\usepackage[T1]{fontenc}
\usepackage{geometry}
\usepackage{hyperref}
\usepackage{listings}
\usepackage{xcolor}
\usepackage{booktabs}
\usepackage{array}
\usepackage{fancyhdr}
\usepackage{enumitem}
\usepackage{float}
\usepackage{mdframed}
\usepackage{longtable}

\geometry{margin=1in}

\definecolor{hydra-blue}{RGB}{5,32,73}
\definecolor{code-bg}{RGB}{248,248,248}
\definecolor{warning-bg}{RGB}{255,243,205}
\definecolor{success-bg}{RGB}{212,237,218}
\definecolor{info-bg}{RGB}{217,237,247}

\hypersetup{colorlinks=true,linkcolor=hydra-blue,urlcolor=blue}

\lstdefinestyle{bash}{backgroundcolor=\color{code-bg},basicstyle=\ttfamily\small,breaklines=true,frame=single,numbers=none}
\lstdefinestyle{js}{backgroundcolor=\color{code-bg},basicstyle=\ttfamily\small,breaklines=true,frame=single,numbers=left,language=Java}

\newmdenv[backgroundcolor=warning-bg,linewidth=0pt,innerleftmargin=10pt,innerrightmargin=10pt,innertopmargin=10pt,innerbottommargin=10pt]{warningbox}
\newmdenv[backgroundcolor=success-bg,linewidth=0pt,innerleftmargin=10pt,innerrightmargin=10pt,innertopmargin=10pt,innerbottommargin=10pt]{successbox}
\newmdenv[backgroundcolor=info-bg,linewidth=0pt,innerleftmargin=10pt,innerrightmargin=10pt,innertopmargin=10pt,innerbottommargin=10pt]{infobox}

\pagestyle{fancy}
\fancyhf{}
\fancyhead[L]{\textbf{Hackathon System}}
\fancyhead[R]{\textbf{Technical Guide}}
\fancyfoot[C]{\thepage}
\setlength{\headheight}{14pt}

\title{\textbf{Hackathon Management \& Voting System}\\[0.5em]\large Technical Documentation}
\author{Computer Science Department\\SUNY New Paltz}
\date{February 2026}

\begin{document}
\maketitle
\tableofcontents
\newpage

%% ============================================================
\section{Overview}
%% ============================================================

The Hackathon Management \& Voting System is a full-stack web application for organizing, judging, and voting on hackathon projects at SUNY New Paltz. It integrates with Hydra SSO for authentication and supports multiple concurrent hackathons.

\begin{infobox}
\textbf{Repository:} \url{https://github.com/compsci-suny-newpaltz/Hackaton-Voting}\\
\textbf{Frontend:} Vue.js 3 + Vite + TailwindCSS\\
\textbf{Backend:} Node.js + Express\\
\textbf{Database:} SQLite (better-sqlite3)\\
\textbf{URL:} \url{https://hydra.newpaltz.edu/hackathons/}
\end{infobox}

%% ============================================================
\section{Architecture}
%% ============================================================

\subsection{Tech Stack}

\begin{table}[H]
\centering
\begin{tabular}{@{}ll@{}}
\toprule
\textbf{Layer} & \textbf{Technology} \\
\midrule
Frontend & Vue.js 3, Vue Router, Vite, TailwindCSS, Axios \\
Backend & Node.js 18, Express 4, better-sqlite3 \\
Auth & Hydra SSO (\texttt{np\_access} JWT cookie) \\
File Upload & Multer (single-file replace model) \\
Deployment & Docker Compose on Hydra (port 45821) \\
\bottomrule
\end{tabular}
\end{table}

\subsection{Project Structure}

\begin{lstlisting}[style=bash]
Hackaton-Voting/
  client/           # Vue.js 3 SPA
    src/
      components/   # Reusable UI components
      views/        # Page-level views
      router/       # Vue Router config
  server/
    index.js        # Express entry point
    db/
      schema.sql    # SQLite schema
      init.js       # Database initialization
    routes/
      hackathons.js # Hackathon CRUD + lifecycle
      projects.js   # Project + team management
      voting.js     # Popular + judge voting
      admin.js      # Admin endpoints
  public/           # Built client assets
  docker-compose.yml
  Dockerfile
\end{lstlisting}

%% ============================================================
\section{Authentication}
%% ============================================================

The system delegates authentication to Hydra SAML Auth via the \texttt{np\_access} JWT cookie.

\begin{enumerate}
    \item User visits \texttt{/hackathons/}
    \item If no valid \texttt{np\_access} cookie, redirect to \texttt{/login?returnTo=/hackathons/}
    \item Hydra performs SAML SSO with Azure AD
    \item On success, Hydra sets \texttt{np\_access} cookie (RS256 JWT)
    \item Hackathon app verifies JWT using Hydra's JWKS endpoint
\end{enumerate}

\subsection{Admin Access}

\begin{itemize}
    \item Hardcoded admin: \texttt{gopeen1@newpaltz.edu}
    \item Faculty auto-admins: users with \texttt{faculty} role in SSO claims
    \item Manual whitelist: additional admins added via admin dashboard
\end{itemize}

%% ============================================================
\section{Hackathon Lifecycle}
%% ============================================================

Each hackathon progresses through 7 states:

\begin{table}[H]
\centering
\begin{tabular}{@{}lll@{}}
\toprule
\textbf{State} & \textbf{Trigger} & \textbf{Description} \\
\midrule
\texttt{upcoming} & Created, start time in future & Not yet accepting submissions \\
\texttt{active} & Current time $\geq$ start time & Accepting project submissions \\
\texttt{ended} & Current time $\geq$ end time & Submissions closed, judging open \\
\texttt{vote\_expired} & Current time $\geq$ vote expiration & All voting closed \\
\texttt{review-period} & Vote expired, review window active & Admin reviews before results \\
\texttt{concluded} & Review period ended & Results publicly visible \\
\texttt{archived} & Admin manually archives & Hidden from default listings \\
\bottomrule
\end{tabular}
\caption{Hackathon state machine}
\end{table}

\begin{infobox}
\textbf{Time validation:} The system enforces \texttt{start\_time < end\_time < vote\_expiration}. The optional review period adds a \texttt{review\_ends\_at} timestamp between vote expiration and public results.
\end{infobox}

%% ============================================================
\section{Voting Systems}
%% ============================================================

\subsection{Popular Vote}

\begin{itemize}
    \item One vote per user per project (enforced by DB uniqueness on \texttt{userId + projectId})
    \item Users cannot vote for their own project
    \item Vote counts displayed live with 5-second polling
    \item Snapshot captured at vote close for historical display
    \item Admin audit page shows all votes per project
\end{itemize}

\subsection{Judge Voting (Rubric System)}

Judges score projects using a customizable category-based rubric:

\begin{table}[H]
\centering
\begin{tabular}{@{}llr@{}}
\toprule
\textbf{Category} & \textbf{Description} & \textbf{Weight} \\
\midrule
Innovation / Creativity & Originality of idea or approach & 1.0 \\
Functionality & Working features, reliability & 1.0 \\
Design / Polish & UX, accessibility, visual quality & 1.0 \\
Presentation / Demo & Communication of idea and goals & 1.0 \\
\bottomrule
\end{tabular}
\caption{Default judge categories (customizable by admin)}
\end{table}

\begin{itemize}
    \item Scores: 1--10 per category with optional comments
    \item Weights: multiplier 0.1--5.0 per category (admin-editable)
    \item Judges can edit scores until judging phase closes
    \item Progress tracking: counter shows \texttt{x/y} projects scored
    \item Saved projects display green outline with checkmark
\end{itemize}

\subsection{Results Calculation}

The weighted score for each project is calculated as:

\[
\text{Score} = \frac{\sum_{i} w_i \cdot s_i}{\sum_{i} w_i}
\]

where $w_i$ is the category weight and $s_i$ is the average judge score for that category.

%% ============================================================
\section{Data Model}
%% ============================================================

\subsection{Core Tables}

\begin{table}[H]
\centering
\begin{tabular}{@{}ll@{}}
\toprule
\textbf{Table} & \textbf{Purpose} \\
\midrule
\texttt{hackathons} & Hackathon metadata, times, settings \\
\texttt{projects} & Team submissions with descriptions, files \\
\texttt{team\_members} & Project-to-email associations \\
\texttt{popular\_votes} & User upvotes (unique per user+project) \\
\texttt{judge\_codes} & Secure access codes for judges \\
\texttt{hackathon\_judge\_categories} & Rubric categories with weights \\
\texttt{judge\_category\_votes} & Per-category scores from judges \\
\texttt{comments} & Project discussion threads \\
\texttt{audit\_log} & Admin action audit trail \\
\bottomrule
\end{tabular}
\caption{Database schema overview}
\end{table}

\subsection{Key Constraints}

\begin{itemize}
    \item \texttt{UNIQUE(judge\_code\_id, project\_id, category\_id)} --- prevents duplicate judge scores
    \item \texttt{CHECK(score >= 1 AND score <= 10)} --- enforces score range
    \item \texttt{UNIQUE(user\_id, project\_id)} on popular votes --- one vote per user
    \item Email domain enforcement: \texttt{@newpaltz.edu} and subdomains only
\end{itemize}

%% ============================================================
\section{API Endpoints}
%% ============================================================

\subsection{Hackathon Management}

\begin{longtable}{@{}lll@{}}
\toprule
\textbf{Method} & \textbf{Path} & \textbf{Description} \\
\midrule
\endhead
GET & \texttt{/hackathons/active} & List active + upcoming hackathons \\
POST & \texttt{/hackathons} & Create hackathon (admin) \\
PUT & \texttt{/hackathons/:id} & Update hackathon (admin) \\
POST & \texttt{/hackathons/:id/archive} & Archive hackathon (admin) \\
\bottomrule
\end{longtable}

\subsection{Project \& Team}

\begin{longtable}{@{}lll@{}}
\toprule
\textbf{Method} & \textbf{Path} & \textbf{Description} \\
\midrule
\endhead
GET & \texttt{/:hid/projects} & List projects \\
POST & \texttt{/:hid/projects} & Create project \\
POST & \texttt{/:hid/projects/:pid/team-members} & Add team member \\
DELETE & \texttt{/:hid/projects/:pid/team-members/:email} & Remove team member \\
\bottomrule
\end{longtable}

\subsection{Voting}

\begin{longtable}{@{}lll@{}}
\toprule
\textbf{Method} & \textbf{Path} & \textbf{Description} \\
\midrule
\endhead
POST & \texttt{/:hid/projects/:pid/vote} & Cast popular vote \\
POST & \texttt{/:hid/judge/:code/vote} & Submit judge scores \\
GET & \texttt{/:hid/results} & Get weighted results \\
\bottomrule
\end{longtable}

\subsection{Admin}

\begin{longtable}{@{}lll@{}}
\toprule
\textbf{Method} & \textbf{Path} & \textbf{Description} \\
\midrule
\endhead
GET & \texttt{/admin/hackathons/:id/categories} & List rubric categories \\
POST & \texttt{/admin/hackathons/:id/categories} & Create category \\
PUT & \texttt{/admin/categories/:id} & Update category \\
DELETE & \texttt{/admin/categories/:id} & Delete category \\
POST & \texttt{/admin/hackathons/:id/categories/reorder} & Reorder categories \\
\bottomrule
\end{longtable}

%% ============================================================
\section{Deployment}
%% ============================================================

\subsection{Docker Compose}

\begin{lstlisting}[style=bash]
# Build and start
cd /home/infra/Hackaton-Voting
docker compose build
docker compose up -d

# Logs
docker logs hackathons-app --tail=50

# Initialize database
docker exec hackathons-app npm run init-db
\end{lstlisting}

\subsection{Local Development}

\begin{lstlisting}[style=bash]
# Install dependencies
npm install
cd client && npm install && cd ..

# Initialize database
npm run init-db

# Create .env from example
cp .env.example .env

# Run dev servers (backend + Vite frontend)
npm run dev
\end{lstlisting}

\begin{warningbox}
\textbf{First admin access:} The hardcoded admin \texttt{gopeen1@newpaltz.edu} and any user with a \texttt{faculty} SSO role are automatically administrators. Access the admin dashboard at \texttt{/hackathons/admin/dashboard}.
\end{warningbox}

%% ============================================================
\section{Security}
%% ============================================================

\begin{itemize}
    \item \textbf{Authentication:} Delegated to Hydra SSO --- no local passwords
    \item \textbf{Email privacy:} Non-team members see masked emails (\texttt{j***e@newpaltz.edu})
    \item \textbf{Judge codes:} Cryptographically secure, hackathon-specific
    \item \textbf{Input validation:} Server-side length limits, domain enforcement, time constraint validation
    \item \textbf{SQL safety:} Parameterized queries via better-sqlite3
    \item \textbf{Audit logging:} All admin actions recorded with timestamp and user
\end{itemize}

%% ============================================================
\section{Implementation Status}
%% ============================================================

\begin{table}[H]
\centering
\begin{tabular}{@{}llc@{}}
\toprule
\textbf{Phase} & \textbf{Description} & \textbf{Status} \\
\midrule
Phase 0--1 & Core constraints, data validation & Complete \\
Phase 2 & Hackathon lifecycle \& concurrency & Complete \\
Phase 3 & Project \& team management & Complete \\
Phase 4 & Voting \& judging overhaul & Complete \\
Phase 5 & UX feedback \& navigation & In Progress \\
Phase 6 & Security \& permissions & Planned \\
Phase 7 & File \& transaction safety & Planned \\
Phase 8 & Performance \& scalability & Planned \\
Phase 9 & Mobile \& responsive & Planned \\
Phase 10 & Documentation & Planned \\
\bottomrule
\end{tabular}
\caption{Development roadmap (see \texttt{TODOS.md} for full details)}
\end{table}

\end{document}
